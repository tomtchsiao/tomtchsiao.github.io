\documentclass[14pt,a4paper,addpoints]{exam}
\usepackage{graphicx}
\usepackage{zhnumber}
\usepackage{ctex}
\usepackage{hyperref}
\usepackage{xcolor}
\usepackage{booktabs}
\usepackage{amsmath, amsthm, amssymb}
\pointname{ 分}
\pointformat{(\thepoints)}
\renewcommand{\thequestion}{\arabic{question}}
\renewcommand{\questionlabel}{\thequestion .}
\renewcommand{\thepartno}{\arabic{partno}}
\renewcommand{\partlabel}{ (\thepartno) }
\renewcommand{\thesubpart}{}
\renewcommand{\subpartlabel}{\thesubpart.}
\renewcommand{\thesubsubpart}{}
\renewcommand{\subsubpartlabel}{\thesubsubpart.}
\renewcommand{\thechoice}{\Alph{choice}}
\renewcommand{\choicelabel}{(\thechoice)}
\renewcommand{\questionshook}{%
    \setlength{\leftmargin}{0pt}%
    \setlength{\labelwidth}{-\labelsep}%
}
\pagestyle{headandfoot}
\firstpagefooter{}{第 \thepage / \numpages 頁}{}
\runningfooter{}{第 \thepage / \numpages 頁}{}
\begin{document}
    \begin{center}
    \fontsize{15pt}{15pt}\selectfont
    政治學期末試卷 \\
    \end{center}
    \vspace{0.3cm}
    \fontsize{14pt}{14pt}\selectfont
    壹、是非題 (3\%, 30 pts) \\
    \begin{questions}
    \question[3] 授課時間為週日。
    \end{questions}
    \vspace{0.3cm}
    貳、解釋題 (5\%, 30 pts) \\
    \begin{questions}
    \question[5] 請解釋何謂政治學。
    \end{questions}
    \vspace{0.3cm}
    參、申論題 (10\%, 30\%, 40 pts) \\
    \begin{questions}
    \question[] 請說明何謂政治制度。
    \begin{parts}
    \part[10] 。
    \part[30] 。
    \end{parts}
    \end{questions}
\end{document}